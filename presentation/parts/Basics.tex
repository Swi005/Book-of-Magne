\subsection{What is a programming language?}
\begin{frame}{What is a programming language?}
    A programming language
    \begin{itemize}[<+->]
        \item is an artificial language(i.e made by us humans on purpose)
        \item is used to tell machines what to do
    \end{itemize}
\end{frame}

\subsubsection{Types of language}
\begin{frame}{Grouping Languages}
    \begin{itemize}[<+->]
        \item Many different ways of grouping
        \item Ex. purpose, typing, paradigm, generality vs. specificity
        \item Look at; generality vs. specificity, and paradigm
    \end{itemize}
\end{frame}

\begin{frame}{Generality vs. Specificity}
    Languages are usually grouped into two categories when based on their specificity
    \begin{itemize}[<+->]
        \item DSLs, small,  targeted at specific problems. Internal/embedded vs external 
        \item GPLs, large, many uses.
    \end{itemize}
\end{frame}

\begin{frame}{Generality vs. Specificity}
    \begin{tabular}{|c|c|c|}%
        \hline
        \textbf{Characteristic} &\textbf{DSL} &\textbf{GPL}\\
        \hline
        \textbf{Domain} &Small, Well-defined & Large, Not specialized\\
        \hline
        \textbf{Size} &Small &Large\\
        \hline
        \textbf{Lifespan} &Weeks, Months &Years, Decades \\
        \hline
    \end{tabular}%
\end{frame}

\begin{frame}{Paradigms}
    \begin{itemize}[<+->]
        \item Imperative Languages
        \item Functional Languages
        \item OOP languages
        \item Logic languages
    \end{itemize}
\end{frame}

\subsection*{Syntax and Semantics}
\begin{frame}{Syntax and Semantics}
    All languages consist of two parts
    \begin{itemize}[<+->]
        \item Syntax
        \item Semantics
    \end{itemize}
\end{frame}

\subsubsection{The Interpreter Process}
\begin{frame}{The Interpreter Process}
    Interpreters differ from Compilers since they execute code instead of translating it.
    \begin{itemize}[<+->]
        \item Lexical Analyzer
        \item Syntax Analyzer
        \item Semantic Analyzer
        \item Evaluator
    \end{itemize}
\end{frame}

\subsection{Meta Programming}
\begin{frame}{Meta Programming}
    A metaprogram is a program that works on \textit{other} programs.\\
    When you see a data structure like BSL or BIPL in Haskell it represents those programs.
\end{frame}

\subsection*{Q\&A}
\begin{frame}{Questions?}
    \begin{figure}
        \centering
        \includegraphics[height = 4.9cm]{intermission/chap04.jpg}
    \end{figure}
\end{frame}