%\section*{LaTeX Steinar}
\subsection*{Teamer}
\begin{frame}{Temaer}
\begin{itemize}
    \item Hver section er noe som vises på Table of contents og øverst som tema
    \item Hver subsection vises i andre rad i denn blåe boksen
    \item bruk stjerne for subsection dersom den ikke skal inn i table of contents
    \item Hver slide skrives inn i \textit{itemize environment}
\end{itemize}
\end{frame}

\subsection*{Eksempel for tabeller}
\begin{frame}{Eksempel}
%en tabell skrives inn i tabular environment
%\begin{center}
\begin{tabular}{l|l|l|l|l}
 StudentNr & Name & Address & KursNr & KursName \\\hline
 580 & Ola NordmaNN & 5075 Bergen Fv 14 & INF237 & Algorithm Engineering\\
 580 & Ola NordmaNN & 5075 Bergen Fv 14 & INF273 & Meta Heuristikker\\
 580 & Ola NordmaNN & 5075 Bergen Fv 14 & INF227 & Logik\\
 256 & Max MustermaNN & 5055 Bergen Lv 85 & INF237 & Algorithm Engineering\\
\end{tabular}
\\[5mm] %avstand fra tabellen
en tabell skrives inn i tabular environment\\
før eller etter tabellen kan det står andre ting, husk å sette avstanden
%\end{center}
\end{frame}

\begin{frame}{Hvordan ha flere tabeller samtidige?}
% svaret er hvill og vfill
\begin{tabular}{l|l|l}
 StudentNr & Name & Address\\\hline
 580 & Ola NordmaNN & 5075 Bergen Fv 14\\
 256 & Max MustermaNN & 5055 Bergen Lv 85\\
\end{tabular}
\vfill
\begin{tabular}{l|l}
KursNr & KursName \\\hline
INF237 & Algorithm Engineering\\
INF273 & Meta Heuristikker\\
INF227 & Logikk\\
\end{tabular}
\hfill
\begin{tabular}{l|l}
 StudentNr & KursNr\\\hline
 580 & INF237\\
 580 & INF273\\
 580 & INF227\\
 256 & INF237\\
\end{tabular}
\end{frame}

% etter hvert tema har vi et bilde av Guillaume som spør noe
% husk å bruke alle Guillaumebilder fra images, slik at folk har den beste andeopplevelsen
% etter hvert section burde det være en slide med "spørsmål" og et Guillaume bilde
\subsection*{Spørretid}
\begin{frame}{Spørsmål?}
    \begin{figure}
        \centering
        \includegraphics[height = 4.9cm]{images/guillaume9.jpg}
        \caption{Guillaume foran Tvindefossen}
        \label{fig:guillaume9}
    \end{figure}
\end{frame}

%=================================
\subsection*{Hva Sander skulle vite}
\begin{frame}{LaTeX informasjoner for Sander}
\begin{itemize}
    \item En frame er teknisk sett ikke det samme som en side. Vi kan innstille at ting etter hvert dykker opp på en slide, da består en side av flere frames, altså forskjellige ting som vises
    \item Den neste sliden forklarer det
\end{itemize}
\end{frame}

\subsection*{Synlighet Lister}
\begin{frame}{Itemize: Automatisk oppdykking av neste punkter}
%% [<+->] en liste dykker opp med et element hver
%% [<+(2)->] skrittlengde er to
\begin{itemize}[<+->]
    \item Itemize funker som i LaTeX
    \item Kommandoen oppe gjør at hver element dykker opp på neste slide
    \item Alt som funker som lister kan dette automatisk
\end{itemize}

\end{frame}

%% <1-> setter på hvilke sub-frames noe skal være synnlig
\begin{frame}{Itemize: Eksplisit synnlighet}
\begin{itemize}
    \item<1-> Dette punktet er synlig fra frame 1 til slutten
    \item<2-> Dette punktet er synlig fra frame 2 til slutten
    \item<2-3> Den også
    \item<3-> Den dykker opp som tredje
    \item<4-> Da er det noe som har forsvunnet!
\end{itemize}

\end{frame}
% \pause Det samme kan gjøres med pause
\begin{frame}{Itemize: Pause}
\begin{itemize}
    \item Det samme kan gjøres med \texttt{pause}
    \item Alt dykker opp til neste pause kommando
    \pause
    \item Også kommer det neste delen til slutten eller neste pausekommandoen
\end{itemize}

\end{frame}

\subsection*{En ting til med pause}
%% Pause kan også brukes for alt på en slide
\begin{frame}{Pause med tekst}
Pause kan også brukes innenfor en stor tekst.\pause~Og alt mulig annet.\pause~Tildesymbolen trengs fordi \LaTeX ellers mangler en space
\end{frame}
%
\begin{frame}{Resten}
Resten er ting som er skrevet på tysk som Lukas var for lat å oversette. Det er mange fine ting blant det, men det kan vi se på senere
\end{frame}

\begin{frame}
\subsection*{Tekstblokker}
    \frametitle{Tekstblokker (da har jeg plutselig oversatt en til)}
    
    Det er tre standardfarger av tekstblokker i \LaTeX. Fargene kan selvfølgelig endres, men ingen vet hvordan.
    
    \begin{block}{Remark}
    Navnene er \texttt{block}, \texttt{alertblock} og \texttt{examples}
    \end{block}
   
    \begin{alertblock}{Important theorem}
    Fransk er et forferdelig språk.
    \end{alertblock}
    
    \begin{examples}
    Un bel avion est un avion qui vole bien.
    \end{examples}
\end{frame}

\begin{frame}[fragile]{Eksempel Python}
\begin{minted}{python}
for i in range(10):
    print("bruk Haskell")
\end{minted}
\end{frame}

% fra her er det bare noen andre tulleting som man ikke trenger i vite men formalt finnes i denne manual

%% Der Visible-Befehl sagt an, auf welchen Folien etwas sichtbar sein soll (analog zu <a-b>)
%\begin{frame}{Der Visible-Befehl}
%Auch der von-bis-Befehl hat für beliebige Sachen ein Pendant. Man kann mit dem \texttt{Visible}-Befehl und geschwungenen Klammern einen Bereich umrahmen, der nur von Teil-Seite x bis y sichtbar sein wird.
%
%\newLine % Der NewLine-Befehl wurde in der Präambel definiert und ist kein LaTeX-Standard. Er führt zu einem angenehmen Zeilenumbruch, der für Zentrierung der Inhalte einer Folie führt
%\visible<2-3>{
% $ G := (V, E)$ with $V$ a set of vertices and $E := \{ (a, b)$ and $(b, a)$ with $a, b \in V $ and $ a \neq b \} $
%
%}
%    
%\visible<3-3>{
%  \newLine
%  $ G = (\{1, 2, 3, 4, 5, 6\}, \{(1, 2), (2, 1),
%      (1, 3), (3, 1),
%      (1, 4), (4, 1), 
%      (2, 4), (4, 2), $ \\
%  \quad\quad $ (2, 5), (5, 2),
%      (2, 6), (6, 2),
%      (3, 4), (4, 3),
%      (3, 6), (6, 3),
%      (5, 6), (6, 5)\}) $
%}
%
%\visible<4>{
%  \begin{center}
%    Ich lasse nun all den Mathekram verschwinden
%  \end{center}
%}
%\end{frame}
%
%
%% Der Only-Befehl funktioniert ähnlich wie visible, gibt jedoch den Platz wieder an das nächste Element zurück
%\begin{frame}{Der Only-Befehl}
%Der Only-Befehl funktioniert wie Visible, aber gibt den Platz wieder frei\\
%\begin{center}
%  \begin{tabular}{cccccc}
%    $ \begin{tabular}{c}1: \\2: \\3: \\4: \\5: \end{tabular} $ &
%    $ \begin{bmatrix}-1.28078 \\0.280776 \\-1 \\-1.28078 \\2.28078 \end{bmatrix} $ &
%    $\rightarrow$ &
%    $ \begin{bmatrix}-1 \\1 \\-1 \\-1 \\1 \end{bmatrix} $ &
%    \visible<2>{$\rightarrow$} &
%    \visible<2>{\noindent\parbox[c]{2.5cm}{ $V_1 = \{1,3,4\}\\V_2 = \{2,5,6\}$}}
%  \end{tabular}\\
%  % Anmerkung: Die Mathematik dieser Folie ist nicht korrekt und dient Anschauungszwecken
%  
%  \only<1>{% Graph 1 ist auf der ersten Folie sichtbar
%    \myGraph
%  }
%  \only<2>{% Graph 2 ist auf der nächsten Folie an seiner Stelle sichtbar
%    \myGraphCorrect
%  }
%\end{center}
%\end{frame}
%
%\section{Sonstiges}
%
%\subsection{Textblöcke}
%
%\subsection{Zweispaltiges Layout}
%\begin{frame}
%    \begin{columns}
%    
%    \column{0.5\textwidth}
%    Darstellung in zwei Spalten ist \textit{natürlich} möglich.
%    $$P\stackrel{?}{=} NP$$
%    \begin{itemize}
%    \item Präsentationsfanatiker können auch hier
%    \item die Spalten nacheinander erscheinen lassen
%    \item Aber wer macht sowas?
%    \end{itemize}
%    
%    \pause
%    \column{0.5\textwidth}
%    Hierfür nutzt man den \texttt{column}-Befehl und setzt die Breite der Columns entsprechend fest. Hier ist das eine 50-50-Aufteilung.
%    \end{columns}
%
%\end{frame}
%
%% Ein komplexeres Beispiel
%% noframenumbering führt dazu, dass eine Seite nicht mit in die Seitenzahlen hinein gezählt wird
%% Wer braucht sowas? Ingen vet
%\begin{frame}[noframenumbering]
% \begin{columns}
%    \begin{column}{0.52\textwidth}
%	\begin{table}
%    	\begin{tabular}{l|l|c|c|}
%    	\multicolumn{2}{c}{}&\multicolumn{2}{c}{Tatsächlich}\\
%    	\cline{3-4}
%    	\multicolumn{2}{c|}{}&Positiv&Negativ\\
%    	\cline{2-4}
%    	\multirow{2}{*}{Vorhergesagt}& Positiv & $TP$ & $FP$\\
%    	\cline{2-4}
%    	& Negativ & $FN$ & $TN$\\
%    	\cline{2-4}
%    	\end{tabular}
%	    \caption{Konfusionsmatrix}
%	    \label{tab:confusion-matrix}
%    \end{table}
%    \end{column}
%    
%    \begin{column}{0.52\textwidth}
%    \begin{center}
%        ${\displaystyle Precision = \frac{TP}{TP+FP} } $\\[5mm]
%        ${\displaystyle Recall = \frac{TP}{TP+FN} } $\\[5mm]
%        ${\displaystyle F_1 = \frac{2\cdot Precision\cdot Recall}{Precision + Recall} } $
%    \end{center}
%    \end{column}
% \end{columns}
%\end{frame}
%
%\subsection{Bilder einfügen}
%\begin{frame}{Graphiken einfügen wie sonst auch}
%    \begin{figure}
%        \centering
%        \includegraphics[height = 4.9cm]{puzzle.png}
%        \caption{Schwarz am Zug, Matt in 5}
%        \label{fig:chesspuzzle}
%    \end{figure}
%\end{frame}