\chapter{Assertions}
Assertions are a statement that checks that some predicate holds.\\
Assertions ensure that a program is always in a valid state.
If for any reason an assertion fails a program crash is the result.
This is usually the preferred option since an invalid state means that the program behaviour is undefined, and can result in security or safety issues.

\section{Types of Assertions}
\subsection*{Pre-Conditions}
Pre-conditions are assertions that must hold before a function or procedure is called.
They are used to ensure that the function is called with valid arguments and that the function can be executed correctly.

\subsubsection*{Post-Conditions}
Post-conditions are assertions that must hold after a function or procedure is called.
They are used to ensure that the function has been executed correctly and that the result is valid.

\subsection*{Invariants}
Invariants are assertions that must hold at all times.
They are used to ensure that the program is always in a valid state.

For example, we can use assertions to make sure that an implementation of the natural numbers is valid.
We could do this by always checking that any argument, value, or result is always greater than or equal to zero.