\section{What is a language}
\subsection{What is a programming language}
    A programming language 
    \begin{itemize}
        \item is an artifical language(i.e made by us humans on purpose)
        \item used to tell machines what to do
    \end{itemize}
    More formally a programming language is a set of rules that converts some input, like strings, into instructions that the computer can follow.
    This is of course a very general description and it, therefore, follows that there are many different types of programming languages.
    IT therefore should come as little surprise that we group languages by features and properties. 
    \subsubsection{Types of languages}
    There are many ways of grouping languages. They can be grouped by Purpose, typing, paradigm, Generality vs. Specificity, and many more.
    For now, we're going to group them by paradigm, and Generality vs. Specificity.
    \subsubsection*{Generality vs. Specificity}
    Languages are usually grouped into two categories when based on their specificity.
    \begin{itemize}
        \item DSL
        \item GPL
    \end{itemize}

    \Gls{DSL} are as the name suggests languages with a "specific" domain.\\
    DSLs usually have limited scope and use. Examples are JSON and SQL.     
    A Domain-Specific Language is a programming language with a higher level of abstraction optimized for a specific class of problems. 
    Optimized for a certain problem/domain. 
    DSLs can be further subdivided into external DSL(separate programming languages), and internal DLS(language-like interface as a library.)

    \Gls{GPL} however are more general and can be used to solve many different problems in many different situations.
    These languages have a wide array
    of uses and are usually what we think of when we hear the words programming language. Examples of GPLs are Java and Haskell.\\
    \begin{figure*}[!h]
        \begin{tabular}{|c|c|c}
            \hline
            \textbf{Characteristic} &\textbf{DSL vs. GPL}\\
            \hline
            \textbf{Domain} & DSLs have a small and well-defined domain\\
            \hline
            \textbf{Size} &GPLs are large, DSLs are usually small\\
            \hline
            \textbf{Lifespan} &GPLs last for years to decades, DSLs typicaly live for shorter periors.\\
            \hline
        \end{tabular}%
        \caption{Some more comparisons between GPLs and DSLs}
    \end{figure*}%

    \subsubsection*{Paradigm}
    We can also classify languages by programming paradigm, some of these are; 
    \begin{itemize}
        \item Imperative Languages, i.e C
        \item Functional Languages, i.e Haskell
        \item Object-oriented Languages, i.e Java, C\#, C++
        \item Logic Languages
    \end{itemize}
    \comment{Write something about this also}
    \comment{Write body}

    \subsubsection*{Syntax and Semantics}
    All programming languages have two parts; the \gls{syntax}, and the \gls{semantics}.\\
    Syntax is the study of \textit{structure}, just as semantics is the study of \textit{meaning}. Or in other words
    the syntax tells us \textit{how} to write legal programs, the semantics tells us \textit{what} those programs do. 

    \subsection{Meta Programming}
    One of the harder things in the course is \gls{meta-programming}. INF222 is usually the first time you've encountered meta-programming and it can be hard a hard concept to grasp. 
    Meta-programming is programming \textit{about} programming. More properly meta-programms treat other programms as data. When you see a datastructure like \gls{BSL} or \gls{BIPL} in Haskell it
    represents a program.