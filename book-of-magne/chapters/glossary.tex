{
\newglossaryentry{DSL}{name={\textbf{Domain Specific Languages}},description={
        A language(i.e not just a library) with abstractions targetet at a specific problem domain.
        \begin{itemize}
            \item \textit{External DSL} - A DSL defined as a seperate programming language.
            \item \textit{Internal/Embedded DSL} - A DSL defined as a language-like interface to a library.
        \end{itemize} 
    }}
\newglossaryentry{GPL}{name={\textbf{General Purpose Languages}},description={
    A language suited for a wide variety of problems and situations but lacks specialized features to deal with specific programs like a DSL.
}}
\newglossaryentry{syntax}{name={\textbf{Syntax}},description={Syntax refers to the rules that define the structure of a language. 
                                                    Syntax in computer programming means the rules that control the structure of the symbols, 
                                                    punctuation, and words of a programming language.}}

\newglossaryentry{semantics}{name={\textbf{Semantics}},description={The semantics of a program concerns the meaning of the program. i.e what it actually does. 
                                                            It can take many forms, sometimes we're only interested in the result of the program.
                                                            Other cases concern the steps taken by the program to reach said output.}}
\newglossaryentry{interpreter}{name={\textbf{Interpreter}},description={An interpreter is a program that directly executes instructions written in a programming language.}}
\newglossaryentry{compiler}{name={\textbf{Compiler}},description={A compiler is a program that translates computer code(\textit{the source language}) into another language(\textit{the target language})
                                                                Compilers usually convert some high level language to some lower level language.}}
\newglossaryentry{AST}{name={\textbf{Abstract Syntax Tree}},description={An Abstract Syntax Tree(AST) is a tree representation of the syntactic structure of our program. 
                                                                Can be represented by using trees or terms, and described by an algebraic data type or regular tree grammar.}}
\newglossaryentry{meta-programming}{name={\textbf{Meta-Programming}},description={Metaprogramming is a programming technique in which computer programs have the ability to treat other programs as their data. 
                                        It means that a program can be designed to read, generate, analyze or transform other programs, and even modify itself while running.}}
\newglossaryentry{BSL}{name={\textbf{Basic Signature Language}},description={Simple language to illustrate signatures}}
\newglossaryentry{BIPL}{name={\textbf{Basic Imperative Programing Language}},description={BIPL is a trivial language to explain basic imperative programming concepts like mutable vars, assignments, control-flow, loops, and iteration}}
\newglossaryentry{expression}{name={\textbf{Expression}},description={An expression is a syntactic construct that can be evaluated in order to obtain its value. The resulting value is usually one of the program's types.}}
\newglossaryentry{wellformed}{name={\textbf{Wellformed}},description={Wellformedness is when a program is following all the rules. smileemoji}}
\newglossaryentry{store}{name={\textbf{Store}},description={Program memory, byte/value array, grows uppwards.}}
\newglossaryentry{env}{name={\textbf{Enviroment}},description={Map describing where things are located in the \gls{store}. Kinda like a phonebook}}
\newglossaryentry{scope}{name={\textbf{Scope}},description={A collection of identifier bindings . i,e what's captured by the eniroment at some point in the code.}}
\newglossaryentry{param}{name={\textbf{Parameter}},description={A parameter is a localvariable that is initialized with the arguments. Also often contains how those args are to be treated.}}
\newglossaryentry{local var}{name={\textbf{Local Variable}},description={A local variable is a variable that only exists within a limited scope.}}
\newglossaryentry{stackframe}{name={\textbf{Stackframe}},description={The stackframe is a snapshot of how the program enviroment looks at a certain point in time. 
                                                                        The stackframe saves things that could be changed by running the procedure and lets us restore the program
                                                                        to the previous state without all the changes made by the procedure.}}
\newglossaryentry{argument}{name={\textbf{Argument}},description={An argument is a value provided to the procedure when it is run. When the procedure is run the parameters of the procedure is initialized with its corresponding argument.}}
\newglossaryentry{copy sem}{name={\textbf{Copy Semantics}},description={Type of argument passing where the parameters are initialized with the value of the arguments}}
\newglossaryentry{ref sem}{name={\textbf{Reference Semantics}},description={Type of argument passing where the parameters point to the the address as the argument.}}
\newglossaryentry{SLE}{name={\textbf{Software Language Engineering}},description={Software Language Engineering is the scientific field that researches language development, and mainteinace of formal descriptios, and tooling of software lanugages}}
\newglossaryentry{BNF}{name={\textbf{Backus-Naur Form}},description={Formal notation for describing grammars. Used to describe the syntax of a language.}}
\newglossaryentry{sum of prod}{name={\textbf{Sum of Products}},description={}}
}
\makenoidxglossaries