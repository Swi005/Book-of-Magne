{
\newglossaryentry{DSL}{name={\textbf{Domain Specific Languages}},description={
        A language(i.e. not just a library) with abstractions targeted at a specific problem domain.
        \begin{itemize}
            \item \textit{External DSL} - A DSL is defined as a separate programming language.
            \item \textit{Internal/Embedded DSL} - A DSL is defined as a language-like interface to a library.
        \end{itemize} 
    }}
\newglossaryentry{GPL}{name={\textbf{General Purpose Languages}},description={
    A language suited for a wide variety of problems and situations but lacks specialized features to deal with specific programs like a DSL.
}}
\newglossaryentry{syntax}{name={\textbf{Syntax}},description={Syntax refers to the rules that define the structure of a language. 
                                                    Syntax in computer programming means the rules that control the structure of the symbols, 
                                                    punctuation, and words of a programming language.}}

\newglossaryentry{semantics}{name={\textbf{Semantics}},description={The semantics of a program concerns the meaning of the program. i.e. what it does. 
                                                            It can take many forms, sometimes we're only interested in the result of the program.
                                                            Other cases concern the steps taken by the program to reach the said output.}}
\newglossaryentry{interpreter}{name={\textbf{Interpreter}},description={An interpreter is a program that directly executes instructions written in a programming language.}}
\newglossaryentry{compiler}{name={\textbf{Compiler}},description={A compiler is a program that translates computer code(\textit{the source language}) into another language(\textit{the target language})
                                                                Compilers usually convert some high-level language to some lower-level language.}}
\newglossaryentry{AST}{name={\textbf{Abstract Syntax Tree}},description={An Abstract Syntax Tree(AST) is a tree representation of the syntactic structure of our program. 
                                                                Can be represented by using trees or terms, and described by an algebraic data type or regular tree grammar.}}
\newglossaryentry{meta-programming}{name={\textbf{Meta-Programming}},description={Metaprogramming is a programming technique in which computer programs treat other programs as their data. 
                                        It means that a program can be designed to read, generate, analyze, or transform other programs, and even modify itself while running.}}
\newglossaryentry{BTL}{name={\textbf{Basic Typed Language}},description={Simple language with a few types and expressions on those types }}
\newglossaryentry{BIPL}{name={\textbf{Basic Imperative Programing Language}},description={An extension of BTL, but with statements and control flow.}}
\newglossaryentry{PIPL}{name={\textbf{Procedural Imperative Programming Language}}, description={An extension of BIPL, but with procedures and functions.}}
\newglossaryentry{expression}{name={\textbf{Expression}},description={An expression is a syntactic construct that can be evaluated in order to obtain its value. The resulting value is usually one of the program's types.}}
\newglossaryentry{wellformed}{name={\textbf{Wellformed}},description={Wellformedness is when a program is following all the rules. smileemoji}}
\newglossaryentry{store}{name={\textbf{Store}},description={Program memory, byte/value array, grows upwards.}}
\newglossaryentry{env}{name={\textbf{Enviroment}},description={Map describing where things are located in the \gls{store}. Kinda like a phonebook}}
\newglossaryentry{scope}{name={\textbf{Scope}},description={A collection of identifier bindings . i.e. what's captured by the environment at some point in the code.}}
\newglossaryentry{param}{name={\textbf{Parameter}},description={A parameter is a local variable that is initialized with the arguments. Also often contains how those args are to be treated.}}
\newglossaryentry{local var}{name={\textbf{Local Variable}},description={A local variable is a variable that only exists within a limited scope.}}
\newglossaryentry{stackframe}{name={\textbf{Stackframe}},description={The stackframe is a snapshot of how the program environment looks at a certain point in time. 
                                                                        The stackframe saves things that could be changed by running the procedure and lets us restore the program
                                                                        to the previous state without all the changes made by the procedure.}}
\newglossaryentry{argument}{name={\textbf{Argument}},description={An argument is a value provided to the procedure when it is run. When the procedure is run the parameters of the procedure are initialized with its corresponding argument.}}
\newglossaryentry{copy sem}{name={\textbf{Copy Semantics}},description={Type of argument passing where the parameters are initialized with the value of the arguments}}
\newglossaryentry{ref sem}{name={\textbf{Reference Semantics}},description={Type of argument passing where the parameters point to the address as the argument.}}
\newglossaryentry{SLE}{name={\textbf{Software Language Engineering}},description={Software Language Engineering is the scientific field that researches language development, and maintenance of formal descriptions, and tooling of software lanugages}}
\newglossaryentry{BNF}{name={\textbf{Backus-Naur Form}},description={Formal notation for describing grammars. Used to describe the syntax of a language.}}
\newglossaryentry{Sum of Products}{name={\textbf{Sum of Products}},description={See chapter 1.3}}
\newglossaryentry{lexical analysis}{name={\textbf{Lexical Analysis}},description={Lexical analysis is the process of converting a sequence of characters into a sequence of tokens.}}
\newglossaryentry{syntax analyzer}{name={\textbf{Syntax Analyzer}},description={Takes a stream of tokens and checks if it follows the rules of the grammar. Outputs a parse tree that is then assembled into an AST}}
\newglossaryentry{semantic analyzer}{name={\textbf{Semantic Analyzer}},description={Takes an AST and checks if it follows the rules of the language. Outputs an annotated AST}}
\newglossaryentry{signature}{name={\textbf{Signature}}, description={A signature defines a set of operations and their types, but not the semantics of the operations.}}
\newglossaryentry{algebra}{name={\textbf{Algebra}}, description={An algebra defines the semantics of a signature.}}
\newglossaryentry{adt}{name={\textbf{Abstract Data Type(ADT)}}, description={An ADT is a data type that is defined by its behavior from the point of view of a user, of the data, specifically in terms of possible values, possible operations on data of this type, and the behavior of these operations.}}

\newglossaryentry{meta-language}{name={\textbf{Meta-Language}}, description={The language a compiler is written in}}
\newglossaryentry{object-language}{name={\textbf{Object Language}}, description={The language that gets compiled}}}
\makenoidxglossaries
